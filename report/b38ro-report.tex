\documentclass{article}

\usepackage[hidelinks]{hyperref}

\usepackage[style=authoryear]{biblatex}
\addbibresource{b38ro-report.bib}
\renewcommand*{\nameyeardelim}{\addcomma\space}

\begin{document}

\title{B38RO - Robotics Group Project}
\author{Muhammad Abban - H00427641 \and Mohammed Laith - H00123456 \and Abdul Maajid Aga - H00123456}
\date{\parbox{\linewidth}{\centering%
  School of Engineering and Physical Sciences \endgraf
  Heriot Watt University Dubai \endgraf\bigskip
  April 2024 }}
\maketitle
\bigskip

\tableofcontents
\newpage

\section{Introduction}

Explain what we are doing with simulating a robotic manipulator, why a robotic manipulator, what we doing with it (a game), and we chose the manipulator.

This document references a book \autocite{craigIntroductionRoboticsMechanics2014} (use in next section actually).

\section{Theory}

Describe stuff about euler angeles (were calculated using \autocite{bernardesQuaternionEulerAngles2022}), FK and IK, and DH params of our robot.

\section{Software}

\subsection{Framework}

mainly talk bout ROS and our joint-angle protocol, as well as our unconventional use of quaternions to store euler angles. also describe and cite our pykin/ikpy library. cite pykin like \autocite{jinPykin2024}

this also talks about most of the project requirements, on how we compute those and whatnot. 

\subsection{Simulation}

talk about how we setup coppeliasim for the simulation, including the arm, its gripper, and the scene.

also a paragraph on how we setup the controller.

\subsection{Game logic}

we describe our TTT AI, our computer vision, and the FSM logic used for playing the game, as well as the pick and place theory.

\subsection{Testing on hardware}

we describe the challenges and considerations we had to take while operating our code on hardware.

\section{Conclusion}

simple stuff + where we our work can be used IRL.

\printbibliography[heading=bibintoc]

\end{document}
